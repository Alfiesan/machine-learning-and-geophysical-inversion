% Mohammad Heriyanto
% mheriyanto37@gmail.com
% mheriyanto.wordpress.com
 
\documentclass[12pt,oneside,a4paper]{article}
\usepackage[top=0.5in, bottom=1.25in, left=1.25in, right=1.25in]{geometry}
\usepackage{hyperref}
\hypersetup{colorlinks=true,urlcolor=blue}

\begin{document}

\title{Machine Learning and Geophysical Inversion}
\author{Mohammad Heriyanto \\web: \href{https://ezygeo.com/}{ezygeo.com} \\ github: \href{https://github.com/ezygeo-ai/machine-learning-and-geophysical-inversion}{github.com/ezygeo-ai} \\~\\ Seri Tutorial Neural Network dan Deep Learning @ 2020 ezygeo.com}
   
\maketitle

\begin{abstract}
This article explain about what is difference between machine learning and traditional geophysical inversion. and application in self-potential dan seismic data.
\end{abstract}

\section{Machine Learning and Geophysical Inversion}
The pupose of this repo is to reconstruct paper from Y. Kim and N. Nakata (The Leading Edge, Volume 37, Issue 12, Dec 2018) about Geophysical inversion versus machine learning in inverse problems and B. Russel (The Leading Edge, Volume 38, Issue 7, Jul 2019) about Machine learning and geophysical inversion — A numerical study. We construct this paper using Python and PyCharm IDE.

Before I will do it, I just try to compare Machine Learning (Multilayer Perceptron Neural Networks (MLPNN)) and Geophysical Inversion (traditional Damped Least Squares (Levenberg–Marquardt) inversion technique (DLS)) using paper from Ilknur Kaftan (Pure Appl. Geophys., Vol. 171, Issue 8, pp 1939–1949, 2014) about Inversion of Self Potential Anomalies with Multilayer Perceptron Neural Networks.

\subsection{Machine Learning and Geophysical Inversion: Self-Potential Case}

Reference: Inversion of Self Potential Anomalies with Multilayer Perceptron Neural Networks Ilknur Kaftan et. al, 2014, Pure Appl. Geophys Syntethic data was created from sphere model with parameters K = 94,686, h = 41.81 m, alpha = 309.37, dan x0 = 77.07 m. This result can be downloaded here and seen below with noise distribution.

\subsubsection{Geophysical Inversion: Damped Least Squares (Levenberg–Marquardt) inversion (DLS)}
I use DLS algorithm from Kode Praktikum GP2103 Metode Komputasi versi Python tutorial: Modul 6 - Metode Komputasi 2018 - GP UP.pdf, page 34 and pm6f code. This result was showed below.

Initial Model x0: 10 | alpha: 100 | h: 10 | K: 94500

Real Model x0: 77.070000 | alpha: 309.370000 | h: 41.810000 | K: 94686.000000

Inversion Model x0: 77.964354 | alpha: 98.834942 | h: 41.029489 | K: -93338.665905

Error: 0.008146

with research paper reference W. Srigutomo, et al, 2016 that is modified below.

\subsubsection{Machine Learning: Multilayer Perceptron Neural Networks (MLPNN)}

Self-Potential Dataset that was generated with 5,000 different spherical models with differing electrical dipole moment, polarization angle, origin and depth to the centre of sphere here and seen below. This dataset contains training (80) dan validation (20) dataset.

\end{document}